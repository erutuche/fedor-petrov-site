\documentclass[12pt,russian]{article}
\usepackage{a4wide}
\usepackage[russian]{babel}
\usepackage{amsfonts}
\usepackage{amssymb, amsthm}
\usepackage{amsmath}
\usepackage[utf8]{inputenc}


%\documentclass[11pt,reqno]{amsart}
\usepackage{amssymb,url, amsthm, amsmath, verbatim}

\begin{document}
\begin{center}
Curriculum Vitae of Fedor Petrov
\end{center}

\textbf{Full Name} Fedor Vladimirovich Petrov

\textbf{Born} 08 Feb. 1983, Leningrad, USSR

\textbf{Citizenship} Russian Federation

\textbf{Employment} Saint-Petersburg Department of the Steklov Institute of Mathematics,
Junior Researcher (since 2007)

\textbf{Diploma} Limit Shapes of Combinatorial Partitions
and Integer Convex Polygons, 2004, Saint-Petersburg State University,
Mathematics and Mechanics Faculty.

\textbf{Thesis} On the Number of Rational Points on Convex Curves and Surfaces, 2007,
The Saint-Petersburg Department of the Steklov Institute of Mathematics

\textbf{Advisor} Anatoly Moiseevich Vershik

\textbf{Fields of Interest} Combinatorics, Geometry of Numbers, Convex Geometry,
Functional Analysis.

\textbf{Teaching experience}

- Teaching high-school students, mostly in 239 Physics
and Mathematics Lyceum Mathematical
Centre in Saint-Petersburg (since 2000).


- Teaching students of the Faculty of Physics of Saint-Petersburg
University (2004,2005,2008). Courses given: Algebra, Analysis
and Topology.

- Teaching students
of the Faculty of Mathematics and Mechanics
of Saint-Petersburg University (since 2010).
Courses given: Mathematical Analysis, Linear Algebra,
Combinatorics, Analytic Number Theory, Lie Groups
and Lie Algebras.

- Educational center of PDMI RAS (since 2007).
Courses given: Probablistic Method, Ramsey Theory,
Additive Combinatorics, Finite Metric Spaces,
Tauberian Theory, Concentration of Measure,
Additional Chapters of Combinatorics.

- Academic University RAS (since 2010).
Courses given: Combinatorics and Graph Theory, Convex Geometry. 

\textbf{Advising}

- Pavel Zatitskiy (Ph.D. in 2014)

\textbf{Editorial board member}

- St. Petersburg Mathematical Journal


\textbf{Awards}

-International Mathematical Olympiad for high
schools students (Bucharest, Romania, 1999) - Gold Medal

- Moebius Contest (hold by Independent University
of Moscow) --- Third Prize for postgraduate students, 2005

- Euler Contest (hold by Euler Foundation, Saint-Petersburg) --- Second Prize for
postgraduate students, 2007

- St. Petersburg Mathematical Society ``Young Mathematician'' award,
2011.

\textbf{Publications}


1. Ф. В. Петров, С.~E.~Рукшин. Две теоремы о выпуклых
многогранниках. {\it Труды Санкт.-Петерб. Мат. Общ.} {\bf 8} (2001)


2. A. Alekseev, F. Petrov. A principle of variations in
  representation theory, in {\it The orbit method in geometry and
  physics} (Marseille, 2000), 1--7, Progr. Math., 213,
  Birkhauser Boston, Boston, MA. (2003)

3. Ю. К. Демьянович, Ф. В. Петров.
Некоторые
тождества и $P$-сплайны. {\it Вестник СПбГУ, сер. мат.
} \textbf{9}, 2, 5--9 (2002)

4. Ф. В. Петров. Предельная форма диаграмм Юнга
для мультипликативных статистик суперполиномиального роста.
  {\it Зап. Научн. Сем. ПОМИ} \textbf{301}, 219--228. (2003)

5. Ф. В. Петров. О количестве рациональных точек на строго выпуклой кривой.
{\it Функ. Анал. Прил.}, {\bf 40}, 1, 30--42. (2006)

6. А. И. Назаров, Ф. В. Петров. О гипотезе С. Л. Табачникова
{\it Алгебра и Анализ}, \textbf{19}, 1, 175-190. (2007)

7. G. Andrews, H. Eriksson, F. Petrov, D. Romik. Integrals, partitions
and MacMahon's theorem. {\it J. Comb. Theor. A.}
\textbf{114}, 3, 545-554. (2007)

8. J. Melleray, F. Petrov, A. Vershik. Espaces m\'{e}triques
lin\'{e}airement rigides. {\it C. R. Math. Acad. Sci.
Paris.} \textbf{344}, 4, 235-240. (2007)


9. Ф. В. Петров. Оценки количества рациональных точек на выпуклых
кривых и поверхностях.
{\it Зап. Научн. Сем. ПОМИ}, {\bf 344}, 174-189. (2007).

10. J.~Melleray, F.~V.~Petrov and  A.~M.~Vershik.
Linearly rigid metric spaces and the embedding problem,
{\it Fund. Math.}, {\bf 199}, No. 2, 177--194. (2008)

11. Ф. В. Петров. Два элементарных подхода к предельным формам диаграмм Юнга.
{\it Зап. Научн. Сем. ПОМИ}, {\bf 370}, 111-131. (2009)



12. F. V. Petrov, D. M. Stolyarov, P. B. Zatitskiy. On embeddings of finite metric spaces in $l_{\infty}^n$.
{\it Mathematika}, {\bf 56}, 1, 135-139. (2010)

13. F. Petrov, A. Vershik. Uncountable Graphs and Invariant Measures on the Set of Universal Countable Graphs. {\it Random Structures and Algorithms},
{\textbf{37}} (3),  389–406. (2010)


14. F. Nazarov, F. Petrov, D. Ryabogin, A. Zvavitch. A remark on the Mahler
conjecture: Local minimality of the unit cube. Duke Math. J. {\textbf {154}}, No. 3, 419-430. (2010)

15. Е. Е. Горячко, Ф. В. Петров. Неразложимые характеры группы рациональных
перекладываний отрезка. {\it Зап. Научн. Сем. ПОМИ} {\bf 378}, 17-31. (2010)

16. Ning Chen, Edith Elkind, Nick Gravin, Fedor Petrov.
Frugal Mechanism Design via Spectral Techniques.
FOCS 2010, proceedings of 51st Annual Symposium on Foundations of Computer Science,
755-764. (2010)

17. Ж. Кассень, Ф. В. Петров, А. Э. Фрид. О возможных скоростях
роста языков Тёплица. {\it Сиб. Мат. Журн.} {\bf 52}, 81-94. (2011)

18. П.~Б.~Затицкий, Ф.~В.~Петров. Об исправлении метрик. {\it Зап. научн. сем. ПОМИ}
\textbf{390}, 201-209. (2011)


19. R. Karasev, F. Petrov. Partitions of nonzero elements of a finite field into pairs.
{\it Isr. J. Math.} {\bf 192}, 143--156. (2012)




20. A.~M.~Vershik, P.~B.~Zatitskiy, F.~V.~Petrov.
Geometry and dynamics of admissible metrics in measure spaces, {\it Cent. Eur. J. Math.} {\bf 11}, 3,
379-400. (2013)

 

21. F.~Petrov, P.~Zusmanovich. On Shirshov bases of graded algebras.
 {\it Isr. J. Math.} {\bf 197}, 23-28. (2013)
 
22. V.~V.~Volkov, F.~V.~Petrov. On the interpolation of integer-valued polynomials.
{\it J. Numb. Theor.}
{\bf 133} (12), 4224–4232. (2013)


23.
А.~М.~Вершик, П.~Б.~Затицкий, Ф.~В.~Петров.
Виртуальная непрерывность измеримых функций многих переменных и теоремы вложения.
Функц. анализ и его прил. {\bf 47} (3), 1–11. (2013)

24. N.~Gravin, F.~Petrov, S.~Robins, D.~Shiryaev.
Convex curves and a Poisson imitation of lattices.
{\it Mathematika} {\bf 60} (1), 139-152. (2014)

25. F.~Petrov. Combinatorial Nullstellensatz approach to polynomial expansion. {\it Acta Arith.} {\bf 165}, 279-282. (2014)

26. А.~М.~Вершик, П.~Б.~Затицкий, Ф.~В.~Петров.
Виртуальная непрерывность измеримых функций многих переменных и ее приложения. {\it Усп. мат. наук} {\bf 69} (6), 81-114. (2014)

27. В.~В.~Волков, Ф.~В.~Петров. Некоторые обобщения теоремы Коши--Дэвенпорта. {\it Зап. научн. сем. ПОМИ} \textbf{432}, 105-110. (2015)

28. G. K\'arolyi, Z.-L. Nagy, F. V. Petrov, V. Volkov. 
A new approach to constant term identities and Selberg-type integrals. {\it Adv. Math.} \textbf{277}, 252-282. (2015)

29. П.~Б.~Затицкий, Ф.~В.~Петров. О субаддитивности масштабирующей энтропийной последовательности. 
{\it Зап. Научн. Сем. ПОМИ} {\bf 436}, 167--173. (2015)


30. А.~М.~Вершик, П.~Б.~Затицкий, Ф.~В.~Петров.
Интегрирование виртуально непрерывных функций по бистохастическим мерам и формула следа ядерных операторов.
{\it Алгебра и Анализ}, \textbf{27} (3), 66--74. (2015)



31. F.~Petrov. Polynomial approach to explicit formulae for generalized binomial coefficients.
{\it Europ. J. of Math.} {\bf 2} (2), pp. 444--458. (2016)



32. Ф.~В.~Петров.  Исправление непрерывных гиперграфов. 
{\it Алгебра и Анализ}, \textbf{28} (6), 84--90. (2016)



33. Ф.~В.~Петров, В.~В.~Соколов. 
Асимптотика жордановой формы случайной нильпотентной матрицы.
{\it Зап. Научн. Сем. ПОМИ}, т. 448, 252-262. (2016)




33. F.~Petrov. Restricted product sets under unique representability, Mosc. J. Comb. Numb. Th.
 \textbf{7}, 1 (2017), published online.

34. F.~Petrov. General parity result and cycle-plus-triangles graphs. J. Graph Th. (2017), published online.

35. J.~Gordon, F.~Petrov. Combinatorics of the Lipschitz polytope. Arnold Math. J. (accepted)

36. F.~Petrov. Combinatorial results implied by many zero divisors in a group ring.
arXiv:1606.03256

\end{document}
